\documentclass[10pt]{article}
\usepackage{amssymb, amsmath}
\pdfpagewidth=8.5truein
\pdfpageheight=11.0truein
\setlength{\textheight}{8.875in}
\setlength{\textwidth}{6.5in}
\setlength{\topmargin}{0in}
\setlength{\headheight}{0in}
\setlength{\headsep}{0in}
\setlength{\parindent}{1pc}
\setlength{\oddsidemargin}{-.1875in}  % Centers text.
\setlength{\evensidemargin}{-.1875in}
\begin{document}

\hfill Eliot Piering
\hfill 9-11-14

\center Homework4

\noindent Problem 1.

\begin{table}[ht]
\caption{Nim game table}
\centering
\begin{tabular}{c c c}
	Marbles &  Strategy & Win or Lose\\ [0.5ex]
\hline
1 & P1 has to take the last marble & lose 
2 & P1 takes 1 marble leaving p2 with 1 marble & win
3 & p1 takes 2 marbles & win
4 & p1 takes 3 marbles & win
5 & p1 takes any amount & lose
6 & p1 takes 1 leaving p2 with 5 marbles & win
7 & p1 takes 2 marbles & win
8 & p1 takes 3 marbles & win
9 & p1 takes any amount & lose
10 & p1 takes 1 marble leaving p2 with 9 marbles & win
11 & p1 takes 2 marbles & win

\end{tabular}
\end{table}

Proposition: P(n): Player 1 will win if the amount of marbles is equal to 4n - 2, 4n - 1, or 4n for all n > 0

Proof:

Base Case:  n = 0; P(1), 
		case 1, 4n - 2 = 2, P1 takes a marble and leaves P2 in the losing position of 1 marble
		case 2, 4n - 1 = 3, P1 takes 2 marbles and leaves P2 in the losing position of 1 marble
		case 3, 4n = 4, P1 takes 3 marbles and leaves P2 in the losing position of 1 marble

Induction Hypothesis:
	P(n) /longrightarrow P(n+1)

Proof:
	Assume P(n) for all piles of size 4n -2, 4n -1, or 4n p1 wins

