\documentclass[10pt]{article}
\usepackage{amssymb, amsmath}
\pdfpagewidth=8.5truein
\pdfpageheight=11.0truein
\setlength{\textheight}{8.875in}
\setlength{\textwidth}{6.5in}
\setlength{\topmargin}{0in}
\setlength{\headheight}{0in}
\setlength{\headsep}{0in}
\setlength{\parindent}{1pc}
\setlength{\oddsidemargin}{-.1875in}  % Centers text.
\setlength{\evensidemargin}{-.1875in}
\begin{document}

\hfill Eliot Piering
\hfill 9-11-14

\center Homework4

\begin{enumerate}
\item	Problem 1.

\begin{table}[ht]
\caption{Nim game table}
\centering
\begin{tabular}{c c c}
	Marbles &  Strategy & Win or Lose \\ [0.5ex]
\hline
1 & P1 has to take the last marble & lose \\
2 & P1 takes 1 marble leaving p2 with 1 marble & win \\
3 & p1 takes 2 marbles & win \\
4 & p1 takes 3 marbles & win \\
5 & p1 takes any amount & lose \\
6 & p1 takes 1 leaving p2 with 5 marbles & win \\
7 & p1 takes 2 marbles & win \\
8 & p1 takes 3 marbles & win \\
9 & p1 takes any amount & lose \\
10 & p1 takes 1 marble leaving p2 with 9 marbles & win \\
11 & p1 takes 2 marbles & win \\

\end{tabular}
\end{table}

Proposition: P(n): Player 1 will win if the amount of marbles is equal to 4n - 2, 4n - 1, or 4n for all n $ > $ 0

Proof:

Base Case:  n = 0; P(1),

		case 1, 4n - 2 = 2, P1 takes a marble and leaves P2 in the losing position of 1 marble

		case 2, 4n - 1 = 3, P1 takes 2 marbles and leaves P2 in the losing position of 1 marble

		case 3, 4n = 4, P1 takes 3 marbles and leaves P2 in the losing position of 1 marble

Induction Hypothesis:

	P(n) $ \longrightarrow $ P(n+1)

Proof:

	Assume P(n) for all piles of size 4n -2, 4n -1, or 4n p1 wins

	case 1, 4(n+1) - 2 == 4n + 4 -2 == 4n + 2 P1 can take away 1 marble leaving 4n + 1, now P2 has to take either 1,2,3 or 5 marbles which would leave 4n, 4n -1 or 4n - 2 marbles so P1 would be in a winning situation
	
	case 2, 4(n+1) - 1 == 4n + 4 -1 == 4n + 1 P1 can take away 2 marbles leaving 4n + 1, now P2 has to take either 1,2,3 or 5 marbles which would leave 4n, 4n -1 or 4n - 2 marbles so P1 would be in a winning situation
	
	case 3,  4(n+1)  == 4n + 4  P1 can take away 3  marbles leaving 4n + 1, now P2 has to take either 1,2,3 or 5 marbles which would leave 4n, 4n -1 or 4n - 2 marbles so P1 would be in a winning situation

So by the induction hypothesis for all piles size 4n -2, 4n - 1, or 4n P1 will win.

\item Problem 2
\begin{enumerate}
\item  \{ x  $ > $ 0 \}  x := x +1 \{ x $ > $ 0 \}

If X is greater than 1 before the program begins, and then you add 1 to x, x will still be greater than 1 when the program ends.

\item \{ x $ > $ 0 \} x := x -1 \{ x $ > $ 0 \}

This program is not correct.  If x = 1 to start the program, then subtracting 1 in the program will leave x to be 0.  0 is not greater than 0 so the final condition will be false

\item \{ x $ < $ 0 \} 

	while x $ \neq $ 0 do

	x := x + 1

	od

	\{ x = 0 \}

	a = x

The loop invariant is a $ \leq $ 0.  

If a $ < $ 0 after the first iteration of the loop a` = a + 1.  a = a` - 1. So a` - 1 $<$ 0, a` $<$ 1; given a is an integer a` $<$ 1 is the same as a` $\leq $ 0.  So the loop invariant is true after 1 iteration of the loop.

Assume a $ < $ 0 and through all iterations of the loop a $ \leq $ 0 prove the postcondition x = 0 when the loop terminates.  This loop will terminate only when a = 0, and because a is always $ \leq $ 0 and each iteration through the loop the value of a moves toward 0, the loop will eventually terminate when a = 0, so the postcondition will be met.    

Given x $ \neq $ 0 and 
\item \{ true \} 

	while x $ \neq $ 0 do

	x := x + 1

	od

	\{ x = 0 \}

This program is incorrect.  If x starts off greater than 0 then this loop will never terminate.  The loop will only terminate when x = 0, howerver if you continue adding 1 to x when x is already greater than 0, x will never equal 0.  Therefore this program is incorrect.
\end{enumerate}

\end{enumerate}
\end{document}
