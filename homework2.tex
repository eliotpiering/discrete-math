\documentclass[10pt]{article}
\usepackage{amssymb, amsmath}
\pdfpagewidth=8.5truein
\pdfpageheight=11.0truein
\setlength{\textheight}{8.875in}
\setlength{\textwidth}{6.5in}
\setlength{\topmargin}{0in}
\setlength{\headheight}{0in}
\setlength{\headsep}{0in}
\setlength{\parindent}{1pc}
\setlength{\oddsidemargin}{-.1875in}  % Centers text.
\setlength{\evensidemargin}{-.1875in}

\begin{document}

\hfill Eliot Piering
\hfill 9-11-14 

\center Homework2

\noindent Problem 1.

\begin{enumerate}
	\item $( P \longrightarrow (Q \vee R))$ given
	\item $( R \longrightarrow S)$ given
	\item $( \neg S \longrightarrow \neg Q)$ given 
	\item $( P )$ given
	\item $( \neg P \vee (Q \vee R) )$ line 1: implies rule
	\item $( \neg R \vee S )$ line 2: imples rule 
	\item $( Q \vee R )$ line 1, line 4: modus ponus
	\item $( S \vee Q )$ line 6, line 7: resolution
	\item $( S \vee \neg Q )$ line 3: implies rule an negation 
\item $( S )$ line 8, line 9: disjunctive syllogism  

\end{enumerate}

\item Problem 2.
\begin{enumerate}
	\item $( \neg P(a) \longrightarrow Q(a) )$	given 
	\item $( P(a) \longrightarrow Q(a) )$		given 
	\item $( \forall x, Q(x) \longrightarrow S(x))$ given 
	\item $( Q(a) \longrightarrow S(x) )$		line 3: universal instansiation 
	\item $( P(a) \longrightarrow S(a) )$		line 1, 2: hypothetical syllogism
	\item $( \neg P(a) \vee Q(a))$			line 2: implies rule
	\item $( P(a) \vee Q(a) )$			line 1: imples rule and double negation
	\item $( Q(a) )$				line 6, 7: disjunctive syllogism (note I think I have the wrong name for this) 
	\item $( S(a) )$				line 4, 9: modus ponens 
\end{enumerate}

\item Problem 3.
\begin{enumerate}
	\item prove $( \neg \forall x P(x) \longrightarrow \forall y (Q(y) \wedge R(y)) )$ is equal to $( \exists x \neg( P(x) \longrightarrow \forall y (Q(y) \wedge R(y))) )$
	\item $( \neg( P(a) \longrightarrow \forall y (Q(y) \wedge R(y))) )$ existential instantiation 
	\item $( \neg( P(a) \longrightarrow (Q(a) \wedge R(a))) )$ universal instanstiation
	\item $(  \neg( \neg P(a) \vee (Q(a) \wedge R(a)) ) )$ implies rule
	\item $( \neg (P(a) \longrightarrow (Q(a) \wedge R(a))) )$ implies rule and double negation
	\item $( \neg (P(a) \longrightarrow forall y (Q(y) \wedge R(y))) )$ universal generalization 
	\item $(  \neg ( \forall x P(x) \longrightarrow forall y (Q(y) \wedge R(y))) )$ universal generalization  
\end{enumerate}

\end{document}
